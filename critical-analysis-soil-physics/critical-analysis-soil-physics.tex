\documentclass[12pt,a4paper]{article}
\usepackage[utf8]{inputenc}
\usepackage[portuguese]{babel}
\usepackage{geometry}
\geometry{margin=2.5cm}
\usepackage{setspace}
\onehalfspacing
\usepackage{amsmath,amsfonts}
\usepackage{graphicx}
\usepackage{hyperref}
\usepackage{fancyhdr}
\usepackage{xcolor}

\usepackage{cite}

% Configurações
\setlength{\headheight}{14pt}
\addtolength{\topmargin}{-2pt}
% Definir a cor verdeagri
\definecolor{verdeagri}{RGB}{34, 139, 34}  % verde floresta
\pagestyle{fancy}
\fancyhf{}
\fancyhead[C]{\small Análise Crítica - Física do Solo e Agricultura 4.0}
\fancyfoot[C]{\thepage}
\renewcommand{\headrulewidth}{0.4pt}

\title{\textbf{Análise Crítica: \\ Aplicação de Sensores de Umidade do Solo de Baixo Custo para 
Manejo de Irrigação em Couve}}
\author{Antoniel Bordin \\ Universidade Tecnologica Federal do Paraná}
\date{2025}

\begin{document}

\maketitle

\begin{abstract}
Esta análise crítica examina o artigo ``Application of low-cost soil moisture sensors for 
irrigation management in Brassica oleracea var. acephala cultivation'' de Pitoro et al. (2025), 
focando na física do solo e na integração de tecnologias da Agricultura 4.0 com foco em sensores 
SKU: CE09640 combinados com arduino e LEDs RGB aplicados na cultura de couve. Avalio 
contribuições em calibração de sensores, curvas de retenção e monitoramento IoT, destacando 
forças, limitações e impacto para agricultores de baixa renda.
\end{abstract}

\section{Introdução}
O cenário global de escassez hídrica e crescimento 
populacional impõe uma pressão sem precedentes sobre 
os sistemas agrícolas, demandando saltos de eficiência 
no uso da água.
Neste contexto, presupoe-se que a física do solo, 
especialmente a dinâmica da umidade, é crucial para a 
agricultura sustentável em cenários de escassez hídrica.
O artigo analisado \cite{pitoro2025} propõe um sistema 
de monitoramento baseado em sensores capacitivos de 
baixo custo (SKU: CE09640) integrado ao Arduino, com 
alertas via LEDs RGB, testado em cultivo de couve. 

\textbf{Tese desta análise:}
Embora o trabalho avance na aplicação acessível de tecnologias 4.0 contendo uma solução inovadora 
e validada para manejo de irrigação, subestima variações edáficas e desafios de escalabilidade, 
limitando sua robustez em solos heterogêneos.

\section{Contribuições para a Física dos Solos}

\subsection{Determinação da Umidade}
O cerne da contribuição do estudo reside na rigorosa calibração 
dos sensores capacitivos SKU: CE09640. O método empregado, 
baseado na perda de massa por secagem em condições ambientais 
controladas, alinha-se com o padrão ouro gravimétrico.
\begin{equation}
\mu = \frac{m_w - m_d}{m_d} \times 100, \quad \theta_v = \mu \cdot \rho_b
\end{equation}
Os resultados da calibração: R $<$ -0.964 (correlação inversa 
forte), R$^2$ $>$ 0.95 (ajuste alto), RMSE $<$ 0.05 (precisão 
aceitável), indicam que, para o Latossolo Vermelho Distrófico 
específico do experimento, o sensor responde de forma previsível 
e precisa às variações de umidade.

\subsection{Relação Teor de Água e Potencial}
Uso de curvas de retenção via modelo van Genuchten ajustado:
\begin{equation}
\theta(\psi_m) = \theta_r + \frac{\theta_s - \theta_r}{[1 + (|\alpha \psi_m|^n)]^m}
\end{equation}
Contendo os seguintes parâmetros (0-0.20 m): $\alpha=1.225$, m=0.260, n=2.918, $\theta_r=0.172$, $\theta_s=0.782$ cm$^3$ cm$^{-3}$, 
compondo uma integração rigorosa de métodos clássicos com sensores acessíveis, enriquecendo Física do Solo.

\section{Contribuições para Agricultura 4.0}
\subsection{Popularização de IoT}
A arquitetura proposta por é composta de Sensores SKU: CE09640 (3.3-5.5V), Arduino Mega 2560 e 
LEDs RGB (azul=saturado $>$ 0.48, verde/amarelo=intermediário 0.28 - 0.48, vermelho=seco $<$ 0.28 
cm$^3$ cm$^{-3}$), com um custo estimado de $<$ R\$100.

\subsection{Democratização e Eficiência}
O uso de LEDs RGB como interface de usuário é um elemento de 
inovação notável no contexto da Agricultura 4.0. Tradicionalmente, 
as tecnologias 4.0 geram fluxos de dados digitais (gráficos, 
números, alertas em apps, entre outros), sendo assim demandam 
certo nível de alfabetização digital e treinamento para interpretação.
Ao traduzir a leitura do sensor e, por consequência, o estado 
hídrico do solo inferido pela física da medida em uma linguagem 
visual universal (cores), o sistema reduz drasticamente a barreira 
de entrada, propiciando assecibilidade à produtores com baixa escolaridade, 
além de proporcionar sustentabilidade para pequenos produtores. WUP: 
IMC 10.95 g L$^{-1}$ (economia 12\% vs ETc 12.42 g L$^{-1}$).

\begin{center}
  \begin{tabular}{|l|c|c|}
  \hline
  \textbf{Método} & \textbf{WUP (g L$^{-1}$)} & \textbf{Economia (\%)} \\
  \hline
  ETc & 12.42 & - \\
  Sensores (IMC) & 10.95 & +12\% \\
  Tensiômetros (IMT) & 9.82 & -  \\
  \hline
  \end{tabular}
\end{center}

\section{Limitações e Desafios para a Escalabilidade}
Nesta revisão avaliamos as limitações apontadas pelo próprio 
estudo no que tange ao contexto (Edáfico, Tecnológico e 
Analítico) e outras inerentes à abordagem:

\begin{enumerate}
    \item \textbf{Variabilidade e Calibração:} A necessidade 
    de calibração específica, embora bem executada, é uma 
    barreira operacional. Para uma adoção ampla, seriam necessários 
    protocolos simplificados de calibração "no campo" ou a 
    disponibilização de bibliotecas de curvas de calibração para 
    diferentes classes de solo.
    \item \textbf{Desempenho em Extremos:} Como observado nos resultados, 
    a precisão do sensor diminui nas faixas de solo muito seco 
    %($\theta < 0,28$ cm$^{³}$ cm$^{⁻³}$) e muito úmido 
    %($\theta > 0,48$ cm$^{³}$ cm$^{⁻³}$), de modo que saturação, a 
    constante dielétrica do meio muda pouco, reduzindo a sensibilidade. Este 
    é um limite físico do método capacitivo que deve ser comunicado aos usuários.
    \item \textbf{Durabilidade e Manutenção:} O estudo foi de curta duração 
    (dois ciclos), entretanto a durabilidade dos sensores de baixo custo, da eletrônica 
    do Arduino e dos LEDs em condições adversas de campo (umidade, temperatura, 
    raios UV) é uma questão em aberto, deste modo a manutenção do sistema, ainda que 
    potencialmente baixa, precisa ser considerada para um periodo maior do que o exposto
    no estudo.
    \item \textbf{Integração com Outras Camadas da 4.0:} O sistema é, em sua 
    essência, uma solução local e isolada. O verdadeiro potencial da Agricultura 4.0 
    estaria na integração desses dados de umidade com outras fontes (imagens 
    de satélite, estações meteorológicas, modelos de previsão) em uma plataforma 
    de gestão, ou seja o protótipo atual é um primeiro passo, mas sua evolução natural 
    seria a transmissão sem fio dos dados para uma dashboard digital.
\end{enumerate}


\section{Conclusão}
O estudo revisado oferece uma contribuição valiosa e prática no campo da interseção 
entre física do solo e tecnologias 4.0. Ele demonstra, com rigor metodológico, 
que é possível desenvolver sistemas de monitoramento de umidade do solo com 
precisão agronômica aceitável utilizando componentes de baixo custo e hardware 
aberto. A inovação da interface visual baseada em LEDs é particularmente 
meritória, pois aborda a questão crítica da usabilidade e inclusão digital.

Do ponto de vista da física do solo, o trabalho reforça a importância da 
calibração \textit{in situ} para qualquer sensor que se proponha a medir 
propriedades do solo de forma indireta. Também evidencia que, para a finalidade 
específica de gestão de irrigação de uma cultura sensível como a couve, a medida 
volumétrica direta (via sensor capacitivo calibrado) pode ser mais eficiente em 
termos de WUP do que a medida do potencial matricial (via tensiômetro) 
em um solo tropical.

Em última análise, este trabalho representa um modelo de como a ciência da 
física do solo pode ser traduzida em tecnologias 4.0 não apenas avançadas, 
mas também acessíveis, inclusivas e orientadas para a solução de problemas 
reais da agricultura sustentável, em sintonia com os desafios globais de 
segurança hídrica e alimentar.


\subsection{Edáficas}

\begin{itemize}
    %Baixa sensibilidade em solo seco/saturado – condições críticas para irrigação.
    \item Solo único: Latossolo Vermelho Distrófico (sand 638, silt 66, clay 296 g kg$^{-1}$)
    \item Calibração local obrigatória (textura, salinidade EC=0.15 dS m$^{-1}$)
    \item Sem testes multi-solo
\end{itemize}

\subsection{Tecnológicas}

\begin{enumerate}
    \item Durabilidade não validada $>$ 1 ano para corrosão
    \item Interface LEDs limitada (sem WiFi/IA)
    \item Amostra sensores pequena (12 calibrados, 4 usados), sem ML para dados
\end{enumerate}

\subsection{Analíticas}
O estudo não propos análise econômica (ROI, manutenção) e possivel comparação com soluções 
comerciais, sendo assim limita à classificação como solução 4.0 completa.

\section{Avaliação Metodológica}
\textbf{Pontos fortes ($\star\star\star\star\star$):} Design fatorial 4x3, comparação ETc/IMT/IMC, DAQ customizado (Arduino mais SD), 2 safras.

\textbf{Pontos fracos ($\star\star$):} Solo único (Latossolo Vermelho), amostra limitada (12 sensores), sem ML para dados.

Sólido, mas poderia ser mais generalista.

\section{Conclusão e Recomendações}
\textbf{Veredito: 4.3/5 $\star\star\star\star$} Democratiza precisão para milhões de produtores.

\textbf{Recomendações:}
\begin{enumerate}
  \item Multi-solo: Testar 5 texturas.
  \item IoT avançado: ESP32 + app.
  \item Automação: Válvulas + ML.
  \item Economia: CBA (3 anos).
\end{enumerate}

\textbf{Frase final:} ``De sensores baratos a irrigação otimizada: revolução acessível.''















O artigo organiza-se em quatro pilares principais:

\begin{enumerate}
    \item \textbf{Calibração Laboratorial}: Sensores calibrados via método gravimétrico (R$^2$ > 0.95, RMSE < 0.05)
    \item \textbf{Curva de Retenção}: Modelo de van Genuchten ($\theta_s$ = 0.782 cm$^3$ cm$^{-3}$)
    \item \textbf{Experimento de Campo}: Design fatorial 4$\times$3 (50-125\% ETc)
    \item \textbf{Resultados}: WUP = 10.95 g L$^{-1}$ (sensores) vs 12.42 g L$^{-1}$ (ETc)
\end{enumerate}

\begin{center}
\begin{tabular}{|l|c|c|c|}
\hline
\textbf{Método} & \textbf{WUP (g L$^{-1}$)} & \textbf{R$^2$} & \textbf{RMSE} \\
\hline
ETc & 12.42 & - & - \\
Sensores (IMC) & 10.95 & 0.95-0.97 & 0.017-0.019 \\
Tensiômetros & 9.82 & - & - \\
\hline
\end{tabular}
\end{center}

\section{Análise das Forças}
\subsection{Física do Solo}
\begin{itemize}
    \item \textbf{Calibração Rigorosa}: Método gravimétrico + van Genuchten para Latossolo Vermelho
    \item \textbf{Equações Precisas}: $\theta_{actual} = \mu \cdot \rho_b$ e $\theta_{rep} = \theta_{fc} - \theta_{actual}$
    \item \textbf{Realismo}: Amostras inalteradas (PVC 10 cm $\times$ 12 cm)
\end{itemize}

\subsection{Agricultura 4.0}
\begin{itemize}
    \item \textbf{IoT Acessível}: Arduino + LEDs RGB (custo < R\$ 100)
    \item \textbf{Interface Intuitiva}: Verde (úmido) → Amarelo → Vermelho (seco)
    \item \textbf{Open Source}: Código disponível em suplemento
\end{itemize}

\section{Críticas e Limitações}
\subsection{1. Física do Solo - Heterogeneidade}
\textbf{Problema}: Calibração em \textbf{um único solo} (64.5\% argila)

\textbf{Evidências}:
\begin{itemize}
    \item Sensores capacitivos falham em solos salinos (EC > 1 dS m$^{-1}$)
    \item Sem testes em Latossolos Arenoso/Álico (comuns no Brasil)
    \item RMSE pode subir para 0.10+ em campo real
\end{itemize}

\subsection{2. Agricultura 4.0 - Escalabilidade}
\begin{itemize}
    \item \textbf{Sem conectividade}: LEDs limitam vs WiFi/LoRaWAN
    \item \textbf{Durabilidade}: Sem testes > 1 ano (corrosão em PVC)
    \item \textbf{Automação}: Falta integração com válvulas solenoidais
\end{itemize}

\subsection{3. Análise Econômica Incompleta}
\begin{itemize}
    \item Custo do sistema: R\$ 85 vs ROI não calculado
    \item Economia de água: 12\% vs custo energia Arduino omitido
\end{itemize}

\section{Conclusão e Recomendações}

\textbf{Vantagens}: Democratiza agricultura de precisão para pequenos produtores.

\textbf{Limitações}: Robustez limitada a solos específicos.

\textbf{Recomendações}:
\begin{enumerate}
    \item \textbf{Multi-solo}: Testar em 3-5 classes texturais
    \item \textbf{IoT Avançado}: ESP32 + MQTT + app Android
    \item \textbf{Automação}: Válvulas + previsão via ML
    \item \textbf{Economia}: Análise CBA (custo-benefício)
\end{enumerate}

\textbf{Veredito Final:} $\star\star\star\star$ (4/5) \\
\textbf{Excelente protótipo, mas precisa de validação ampliada.}

\begin{thebibliography}{9}
\bibitem{pitoro2025}
Pitoro, V.S.J. et al. (2025). Application of low-cost soil moisture sensors for irrigation management in Brassica oleracea var. acephala cultivation. 
{\em Smart Agricultural Technology}, {\bf 12}, 101596.

\bibitem{ahmed2018}
Ahmed, N. et al. (2018). IoT for Smart Precision Agriculture. 
{\em IEEE Internet of Things Journal}.

\bibitem{dobriyal2012}
Dobriyal, P. et al. (2012). Review of soil moisture estimation methods. 
{\em Journal of Hydrology}.
\end{thebibliography}

\end{document}